\documentclass[12pt]{article}
\usepackage{graphicx} % Required for inserting images
\usepackage{amsfonts}
\usepackage{enumitem}
\usepackage{amsthm}
\usepackage{mathabx}
\usepackage{latexsym}
\usepackage{amsmath}

\title{Math 115A Homework 5}
\author{Jaden Ho}
\date{November 20, 2024}

\begin{document}

\maketitle
\section{Introduction}


\begin{enumerate}
    \item Problem 2.5 3e) 
    $\beta = \{x^2-x, x^2+1, x-1 \}$ and $\beta' = \{5x^2-2x -3, -2x^2+5x+5, 2x^2-x-3 \}$
    
    $5x^2 -2x - 3 = 5(x^2 - x) + 0(x^2 + 1) + 3(x-1) \xrightarrow{} \begin{pmatrix}
        5 \\
        0 \\
        3 \\
    \end{pmatrix}$

    $-2x^2+5x+5 = -6(x^2 - x) + 4(x^2 + 1) + 1(x-1) \xrightarrow{} \begin{pmatrix}
        -6 \\
        4 \\
        -1 \\
    \end{pmatrix}$

    $2x^2-x-3 = 3(x^2 - x) + -1(x^2 + 1) + 2(x-1) \xrightarrow{} \begin{pmatrix}
        3 \\
        -1 \\
        2 \\
    \end{pmatrix}$

    $[Id]_{\beta'}^{\beta} = \begin{pmatrix}
        5 & -6 & 3 \\
        0 & 4 & -1 \\
        3 & -1 & 2 \\
    \end{pmatrix}$
    
    \item Problem 2.5 4) Let T be the linear operator on $\mathbb{R^2}$ defined by $T(\begin{pmatrix}
        a\\
        b \\
    \end{pmatrix}) = \begin{pmatrix}
        2a + b \\
        a - 3b
    \end{pmatrix}$, let $\beta$ be the standard ordered basis for $R^2$, and let $\beta' = \{\begin{pmatrix}
        1\\
        1 \\
    \end{pmatrix}, \begin{pmatrix}
        1\\
        2 \\
    \end{pmatrix}\}$. Use Theorem 2.23 and the fact that $\begin{pmatrix}
        1 & 1\\
        1 & 2 \\
    \end{pmatrix}^{-1} = \begin{pmatrix}
        2 & -1\\
        -1 & 1 \\
    \end{pmatrix}$ 
    to find $[T]_{\beta'}$. 
    
    $\beta = \{\begin{pmatrix}
        1\\
        0 \\
    \end{pmatrix}, \begin{pmatrix}
        0\\
        1 \\
    \end{pmatrix}\}$

    $\beta' = \{\begin{pmatrix}
        1\\
        1 \\
    \end{pmatrix}, \begin{pmatrix}
        1\\
        2 \\
    \end{pmatrix}\}$

    T($\begin{pmatrix}
        1\\
        0 \\
    \end{pmatrix}$) = $\begin{pmatrix}
        2 * 1+ 0\\
        1 - 3 * 0 \\
    \end{pmatrix} = \begin{pmatrix}
        2\\
        1\\
    \end{pmatrix} = 2 \begin{pmatrix}
        1\\
        0\\
    \end{pmatrix} + 1\begin{pmatrix}
        0\\
        1\\
    \end{pmatrix}$ : $\begin{pmatrix}
        2\\
        1\\
    \end{pmatrix}$

     T($\begin{pmatrix}
        0\\
        1 \\
    \end{pmatrix}$) = $\begin{pmatrix}
        2 * 0 + 1\\
        0 - 3 * 1 \\
    \end{pmatrix} = \begin{pmatrix}
        2\\
        -3\\
    \end{pmatrix} = 2 \begin{pmatrix}
        1\\
        0\\
    \end{pmatrix} + -3\begin{pmatrix}
        0\\
        1\\
    \end{pmatrix} = \begin{pmatrix}
        1\\
        -3\\
    \end{pmatrix}$

    $[T]_\beta = \begin{pmatrix}
        2 & 1 \\
        1 & -3 \\
    \end{pmatrix}$

    Theorem 2.23 states that:
    \begin{equation}
        [T]_{\beta'} = Q^{-1}[T]_{\beta}Q
    \end{equation}

    \begin{equation}
        [T]_{\beta'} = \begin{pmatrix}
        2 & -1\\
        -1 & 1 \\
    \end{pmatrix} \begin{pmatrix}
        2 & 1\\
        1 & -3 \\
    \end{pmatrix} \begin{pmatrix}
        1 & 1 \\
        1 & 2 \\
    \end{pmatrix}  
    \end{equation}

    \begin{equation}
        [T]_{\beta'} = \begin{pmatrix}
            2 * 2 + -1(1) & 2 * 1 + -1 * -3 \\
            -1 * 2 + 1 * 1 & -1 * 1+ 1 * -3 \\
        \end{pmatrix} \begin{pmatrix}
        1 & 1 \\
        1 & 2 \\
    \end{pmatrix}    
    \end{equation}

    \begin{equation}
        [T]_{\beta'} = \begin{pmatrix}
            3 & 5\\
            -1 & -4 \\
        \end{pmatrix} \begin{pmatrix}
        1 & 1 \\
        1 & 2 \\
    \end{pmatrix} = \begin{pmatrix}
        3 * 1 + 5 * 1 & 3 * 1 + 5 * 2 \\
        -1* 1 + -4 * 1 & -1* 1+ -4 * 2 \\
    \end{pmatrix} 
    \end{equation}

    $[T]_{\beta'} = \begin{pmatrix}
        8 & 13 \\
        -5 & -9 \\
    \end{pmatrix}$
    
    
    \item Problem 2.5 6b)

    [$L_A$]$_\beta$ = $\begin{pmatrix}
        1 & 2 \\
        2 & 1
    \end{pmatrix} \begin{pmatrix}
        1 \\
        1 \\
    \end{pmatrix} = \begin{pmatrix}
        1 + 2 \\
        2 + 1 \\
    \end{pmatrix} = \begin{pmatrix}
        3 \\
        3 \\
    \end{pmatrix} = 3\begin{pmatrix}
        1 \\
        1 \\
    \end{pmatrix} \rightarrow \begin{pmatrix}
        3 \\
        0 \\
    \end{pmatrix}$ 

    
    [$L_A$]$_\beta$ = $\begin{pmatrix}
        1 & 2 \\
        2 & 1
    \end{pmatrix} \begin{pmatrix}
        1 \\
        -1 \\
    \end{pmatrix} = \begin{pmatrix}
        1 - 2 \\
        2 - 1 \\
    \end{pmatrix} = \begin{pmatrix}
        1 \\
        -1 \\
    \end{pmatrix} = -1\begin{pmatrix}
        1 \\
        -1 \\
    \end{pmatrix}$ 

    [L$_A$]$_\beta$ = $\begin{pmatrix}
        3 & 0 \\
        0 & -1\\
    \end{pmatrix}$

    Q will just be the matrix form of the ordered basis vectors of $\beta$: \\
    Q = $\begin{pmatrix}
        1 & 1 \\
        1 & -1 \\
    \end{pmatrix}$
    
    \item 5.1 Problem 2a) T$\begin{pmatrix}
        a \\
        b 
    \end{pmatrix} = \begin{pmatrix}
        2a - b \\
        5a + 3b
    \end{pmatrix}$

    If we put T into matrix form we get: $\begin{pmatrix}
        2 & -1 \\
        5 & 3
    \end{pmatrix}$
    
    det(T) = (2 * 3) - (-1 * 5) = 11

    $\begin{pmatrix}
        2 - \lambda & -1 \\
        5 & 3 - \lambda
    \end{pmatrix} = (2 - \lambda)(3-\lambda) -(-1 * 5) = 6 - 2\lambda - 3\lambda + \lambda^2 + 5$ \\ = 
    $\lambda^2 - 5\lambda + 11$


    \item 5.1 Problem 3f) 
    V = $M_{2 \times 2}(R)$, T($\begin{pmatrix}
        -7a - 4b + 4c - 4d & b \\
        -8a - 4b + 5c - 4d & d\\
    \end{pmatrix}$)

    Let $\beta = \{ \begin{pmatrix}
        1 & 0 \\
        1 & 0 \\
    \end{pmatrix}, \begin{pmatrix}
        -1 & 2 \\
        1 & 0 \\
    \end{pmatrix}, \begin{pmatrix}
        1 & 0 \\
        2 & 0 \\
    \end{pmatrix}, \begin{pmatrix}
        -1 & 0 \\
        0 & 2 \\
    \end{pmatrix} \}$
    
    T($\begin{pmatrix}
        1 & 0 \\
        1 & 0 \\
    \end{pmatrix}$) = $\begin{pmatrix}
        -7*1 + 0 + 4*1 - 0 & 0 \\
        -8*1 + 0 + 5*1 - 0 & 0
    \end{pmatrix} = \begin{pmatrix}
        -3 & 0 \\
        -3 & 0
    \end{pmatrix} = -3\begin{pmatrix}
        -3 & 0 \\
        -3 & 0
    \end{pmatrix} + 0 + 0 + 0 \rightarrow \begin{pmatrix}
        -3 \\
        0 \\
        0\\
        0\\
    \end{pmatrix}$

    T($\begin{pmatrix}
        -1 & 2 \\
        1 & 0 \\
    \end{pmatrix}$) = $\begin{pmatrix}
        -1*1 + 0 + 0 + 0 & 2 \\
        8  - 4(2) + 0 & 0
    \end{pmatrix} = \begin{pmatrix}
        -1 & 2 \\
        0 & 0
    \end{pmatrix} = 1\begin{pmatrix}
        -1 & 2 \\
        0 & 0
    \end{pmatrix} \rightarrow \begin{pmatrix}
        0 \\
        1 \\
        0\\
        0\\
    \end{pmatrix}$

    T($\begin{pmatrix}
        1 & 0 \\
        2 & 0 \\
    \end{pmatrix}$) = $\begin{pmatrix}
        -7*1 + 4(2) & 0 \\
        -8(1) + 5(2) & 0
    \end{pmatrix} = \begin{pmatrix}
        1 & 0 \\
        2 & 0
    \end{pmatrix} = 1\begin{pmatrix}
        1 & 0 \\
        2 & 0
    \end{pmatrix} \rightarrow \begin{pmatrix}
        0 \\
        0 \\
        1\\
        0\\
    \end{pmatrix}$

    T($\begin{pmatrix}
        -1 & 0 \\
        0 & 2 \\
    \end{pmatrix}$) = $\begin{pmatrix}
        -7(-1) + -4(2)  & 0 \\
        -8(-1) - 4(2) & 2
    \end{pmatrix} = \begin{pmatrix}
        -1 & 0 \\
        0 & 2
    \end{pmatrix} = 1\begin{pmatrix}
        -1 & 0 \\
        0 & 2
    \end{pmatrix} \rightarrow \begin{pmatrix}
        0 \\
        0 \\
        0\\
        1\\
    \end{pmatrix}$ 

    [T]$_\beta = \begin{pmatrix}
        -3 & 0 & 0 & 0 \\
        0 & 1 & 0 & 0 \\
        0 & 0 & 1 & 0 \\
        0 & 0 & 0 & 1\\
    \end{pmatrix}$

    Since [T]$_\beta$ is a diagonal matrix, the eigenvalues of T are $-3, 1, 1, 1$. 

    $\beta$ is a basis consisting of eigenvectors of T. 

    \item 5.1 Problem 4cd)
        \begin{enumerate}[label=(\alph*)]
            \item[c)] $A = \begin{pmatrix}
                i & 1 \\
                2 & -i \\
            \end{pmatrix}$ for F = C. 

            $det(A - \lambda I) = det\begin{pmatrix}
                i - \lambda & 1 \\
                2 & -i - \lambda \\
            \end{pmatrix}$
            = $(i - \lambda)(-i - \lambda) - (2) = 0 = 1 - \lambda i + \lambda i + \lambda^2= \lambda^2 - 1 = 0$
            
            $\lambda = -1, 1$ 

            For $\lambda  = -1$: 

            $\begin{pmatrix}
                i - (-1) & 1 \\
                2 & -i - (-1) \\
            \end{pmatrix} = \begin{pmatrix}
                i + 1 & 1 \\
                2 & -i + 1 \\
            \end{pmatrix}$

            $\begin{pmatrix}
                i + 1 & 1 & 0 \\
                2 & -i + 1 & 0 \\
            \end{pmatrix}$

            rank = 1, so gemu = 1

            $(i + 1)x + y = 0$ \\
            $2x + (-i + 1)y = 0$ \\ 

            x = $-\frac{1}{2} + \frac{i}{2}$ \\
            y = 1
            
            $E_{-1} =  span(\begin{pmatrix}
                -\frac{1}{2} + \frac{i}{2} \\
                1 \\
            \end{pmatrix})$

             For $\lambda  = 1$: 

            $\begin{pmatrix}
                i - (1) & 1 \\
                2 & -i - (1) \\
            \end{pmatrix} = \begin{pmatrix}
                i - 1 & 1 \\
                2 & -i - 1 \\
            \end{pmatrix}$

            $\begin{pmatrix}
                i - 1 & 1 & 0 \\
                2 & -i - 1 & 0 \\
            \end{pmatrix}$

            rank = 1, so gemu = 1

            $(i - 1)x + y = 0$ \\
            $2x + (-i - 1)y = 0$ \\ 

            x = $\frac{1}{2} + \frac{i}{2}$ \\
            y = 1
            
            $E_1 = span(\begin{pmatrix}
                \frac{1}{2} + \frac{i}{2} \\
                1 \\
            \end{pmatrix})$

            Q = $\begin{pmatrix}
                -\frac{1}{2} + \frac{i}{2} & \frac{1}{2} + \frac{i}{2} \\
                1 & 1
            \end{pmatrix}$

            basis = $ \{\begin{pmatrix}
                -\frac{1}{2} + \frac{i}{2} \\
                1 \\
            \end{pmatrix}, \begin{pmatrix}
                \frac{1}{2} + \frac{i}{2} \\
                1 \\
            \end{pmatrix} \}$
            
            D = $Q^{-1}AQ$ = $\begin{pmatrix}
                -1 & 0 \\
                0 & 1 \\
            \end{pmatrix}$

        \item[d)] A = $\begin{pmatrix}
            2 & 0 & -1 \\
            4 & 1 & -4\\
            2 & 0 & -1 \\
        \end{pmatrix}$

        det($A - \lambda I$) = $\begin{pmatrix}
            2 - \lambda & 0 & -1 \\
            4 & 1 - \lambda & -4\\
            2 & 0 & -1 - \lambda\\
        \end{pmatrix}$ = \\ $0 + (-1)^{2 + 2}(1 - \lambda)det(\begin{pmatrix}
            2 - \lambda & -1 \\
            2 & -\lambda - 1
        \end{pmatrix}) + 0$

        
        $= (1 - \lambda)((2 - \lambda)(-\lambda - 1) - (-1 * 2)) = \lambda^2 - \lambda = 0$

        Eigenvalues: $\lambda = 0, 1$

        $\lambda = 0$: 
        
        $\begin{pmatrix}
            2 & 0 & -1 \\
            4 & 1 & -4\\
            2 & 0 & -1 \\
        \end{pmatrix} = \begin{pmatrix}
            2 & 0 & -1 \\
            4 & 1 & -4\\
            0 & 0 & 0\\
        \end{pmatrix} = \begin{pmatrix}
            1 & 0 & -\frac{1}{2} \\
            0 & 1 & -2\\
            0 & 0 & 0\\
        \end{pmatrix}$

        rank = 2 so dim(ker(A)) = gemu = 1
        
        x - $-\frac{1}{2}$ = 0 \\
        y - 2z = 0 \\
        z is free

        Let z = 1

         basis for $E_0$ = span($\begin{pmatrix}
            \frac{1}{2} \\
            2 \\
            1 \\
        \end{pmatrix})$

    $\lambda = 1$: 
        $\begin{pmatrix}
            2 - 1 & 0 & -1 \\
            4 & 1 - 1 & -4\\
            2 & 0 & -1 - 1\\
        \end{pmatrix} = \begin{pmatrix}
            1 & 0 & -1 \\
            4 & 0 & -4\\
            2 & 0 & -2\\
        \end{pmatrix} = \begin{pmatrix}
            1 & 0 & -1 \\
            0 & 0 & 0\\
            0 & 0 & 0\\
        \end{pmatrix}$
        
        rank = 1 so dim(ker(A)) = gemu = 2

        x - z = 0

        y, z are free 

        basis for $E_1$ = span($\begin{pmatrix}
            0\\
            1 \\
            0 \\
        \end{pmatrix}, \begin{pmatrix}
            1\\
            0 \\
            1 \\
        \end{pmatrix}$)

        basis = $\{\begin{pmatrix}
            \frac{1}{2} \\
            2 \\
            1 \\
        \end{pmatrix}, \begin{pmatrix}
            0\\
            1 \\
            0 \\
        \end{pmatrix}, \begin{pmatrix}
            1\\
            0 \\
            1 \\
        \end{pmatrix} \}$

        Q = $\begin{pmatrix}
            \frac{1}{2} & 0 & 1 \\
            2 & 1 & 0 \\
            1 & 0 & 1
        \end{pmatrix}$

        D = $Q^{-1}AQ = \begin{pmatrix}
            0 & 0 & 0 \\
            0 & 1 & 0 \\
            0 & 0 & 1
        \end{pmatrix}$
            
        \end{enumerate}

    \item 5.1 Problem 5ae)
        \begin{enumerate}[label=(\alph*)]
            \item[a)] 
            $V = \mathbb{R}^2 $ and T(a,b) = $(-2a + 3b, -10a + 9b)$

            If we put T(a,b) in matrix form: 
            
            Let A = $\begin{pmatrix}
                -2 & 3 \\
                -10 & 9 \\
            \end{pmatrix}$

            To compute the eigenvalues we must put A into the form $(A - \lambda I)v = 0$

            \begin{equation}
                \begin{pmatrix}
                -2 - \lambda & 3 \\
                -10 & 9 - \lambda \\
            \end{pmatrix}
            = (-2-\lambda)(9 - \lambda) - 3(-10) = 0
            \end{equation}

            \begin{equation}
                -18 + 2\lambda - 9\lambda + \lambda^2 + 30 = 0 
            \end{equation}

            \begin{equation}
                \lambda^2 - 7\lambda + 12 = 0 
            \end{equation}

            \begin{equation}
                (\lambda - 3)(\lambda - 4) = 0
            \end{equation}

            $\lambda = 3, 4$

            $\lambda = 3$:
            \begin{equation}
                \begin{pmatrix}
                -2 - 3 & 3 \\
                -10 & 9 - 3 \\
                \end{pmatrix} = \begin{pmatrix}
                -5 & 3 \\
                -10 & 6 \\
                \end{pmatrix}
            \end{equation}

            $-5x + 3y = 0$ \\
            $-10x + 6y = 0$ \\

            $x = \frac{1}{5}$, $y = \frac{1}{3} \xrightarrow{} \begin{pmatrix}
                \frac{1}{5} \\
                \frac{1}{3}
            \end{pmatrix}$ 

            $\lambda = 4$:
            \begin{equation}
                \begin{pmatrix}
                -2 - 4 & 3 \\
                -10 & 9 - 4 \\
                \end{pmatrix} = \begin{pmatrix}
                -6 & 3 \\
                -10 & 5 \\
                \end{pmatrix}
            \end{equation}

            $-6x + 3y = 0$ \\
            $-10x + 5y = 0$ \\

            $x = 1$, $y = 2 \xrightarrow{} \begin{pmatrix}
                1 \\
                2 \\
            \end{pmatrix}$

            \begin{equation}
                \begin{pmatrix}
                    \frac{1}{5} & 1 \\
                    \frac{1}{3} & 2 \\
                \end{pmatrix}
            \end{equation}

            $\beta = \{(\frac{1}{5}, \frac{1}{3}), (1,2) \}$
            
            \item[e)] $V = P_2(R)$ and $T(f(x)) = xf'(x) + f(2)x + f(3)$
            $T(1) = 1 * 0 + 1 * x + 1 = 1 + x$

            $T(x) = x * 1 + 2x + 3 = 3 + 3x$

            $T(x^2) = x * 2x + 4x + 9 = 9 + 4x + 2x^2$

            $[T]_\beta = \begin{pmatrix}
                1 & 3 & 9 \\
                1 & 3 & 4 \\
                0 & 0 & 2 \\
            \end{pmatrix}$ 

            To find eigenvalues, we put it into the form; $(A - I\lambda)v = 0$: 

            det $\begin{pmatrix}
                1 - \lambda & 3 & 9 \\
                1 & 3 - \lambda & 4 \\
                0 & 0 & 2 - \lambda\\
            \end{pmatrix}$ = 0
        \end{enumerate}

        LHS:
        \begin{equation}
            det(A - I\lambda) = 0 - 0 + (2 - \lambda) * det(\begin{pmatrix}
                1 - \lambda & 3 \\
                1 & 3 - \lambda \\
            \end{pmatrix} 
        \end{equation}

        \begin{equation}
            (2 - \lambda) * (1 - \lambda)(3 - \lambda) - 3(1) = (2 - \lambda)(3 - \lambda - 3\lambda + \lambda^2 - 3)
        \end{equation}

        \begin{equation}
            (2 - \lambda)(\lambda^2 - 4\lambda) = (2 - \lambda)(\lambda)(\lambda - 4) = 0
        \end{equation}

        $\lambda = 0, 2, 4$

        At $\lambda = 0$: 
            $\begin{pmatrix}
                1 - 0 & 3 & 9 \\
                1 & 3 - 0 & 4 \\
                0 & 0 & 2 - 0\\
            \end{pmatrix}$ = $\begin{pmatrix}
                1 & 3 & 9 \\
                1 & 3 & 4 \\
                0 & 0 & 2\\
            \end{pmatrix}$

            x + 3y + 9z = 0 \\
            x + 3x + 4z = 0 \\ 
            2z = 0 \\

            z = 0, x + 3y = 0 \\
            x = -3y 

            $\begin{pmatrix}
                -3 \\
                1 \\
                0 \\
            \end{pmatrix}$

        At $\lambda = 2$:
        $\begin{pmatrix}
                1 - 2 & 3 & 9 \\
                1 & 3 - 2 & 4 \\
                0 & 0 & 2 - 2\\
            \end{pmatrix}$ = $\begin{pmatrix}
                -1 & 3 & 9 \\
                1 & 1 & 4 \\
                0 & 0 & 0\\
            \end{pmatrix}$

            $\begin{pmatrix}
                -1 & 3 & 9 & 0\\
                1 & 1 & 4 & 0\\
                0 & 0 & 0 & 0\\
            \end{pmatrix}$ -$R_1 \xrightarrow{} R_1$ $\begin{pmatrix}
                1 & -3 & -9 & 0\\
                1 & 1 & 4 & 0\\
                0 & 0 & 0 & 0\\
            \end{pmatrix} R_2 - R_1 \xrightarrow{} R_2$ $\begin{pmatrix}
                1 & -3 & -9 & 0\\
                0 & 4 & 13 & 0\\
                0 & 0 & 0 & 0\\
            \end{pmatrix}$ $\frac{1}{4}R_2 \xrightarrow{} R_2 \begin{pmatrix}
                1 & -3 & -9 & 0 \\
                0 & 1 & \frac{13}{4} & 0 \\
                0 & 0 & 0 & 0
            \end{pmatrix}$ $R_1 + 3R_2 \xrightarrow{} R_1$ $\begin{pmatrix}
                1 & 0 & \frac{3}{4} & 0\\
                0 & 1 & \frac{13}{4} & 0\\
                0 & 0 & 0 & 0\\
            \end{pmatrix}$

            $x = -3/4t, y = -13/4t, z = t$
            
            If $z = 4$, then x = -3, y = -13. 
            
            $\begin{pmatrix}
                -3 \\
                -13\\
                4 \\
            \end{pmatrix}$

        At $\lambda = 4$: 
        $\begin{pmatrix}
                1 - 4 & 3 & 9 \\
                1 & 3 - 4 & 4 \\
                0 & 0 & 2 - 4\\
            \end{pmatrix}$ = $\begin{pmatrix}
                -3 & 3 & 9 \\
                1 & -1 & 4 \\
                0 & 0 & -2\\
            \end{pmatrix}$

        $\begin{pmatrix}
                -3 & 3 & 9 & 0\\
                1 & -1 & 4 & 0\\
                0 & 0 & -2 & 0\\
            \end{pmatrix}$

        -2z = 0, z = 0

        -3x + 3y = 0 \\
        x - y = 0 \\

        x = 1, y = 1

        $\begin{pmatrix}
            1 \\
            1 \\
            0\\
        \end{pmatrix}$

        Matrix: 
        $\begin{pmatrix}
            -3 & -3 & 1 \\
            1 & -13 & 1\\
            0 & 4 & 0\\
        \end{pmatrix}$
        
        $\beta = \{-3 + x, -3 - 13x + 4x^2, 1 + x\}$
    \item 5.1 Problem 9ab
    \begin{enumerate}[label=(\alph*)]
        \item Prove that a linear operator T on a finite-dimensional vector space is invertible if and only if zero is not an eigenvalue of T. 

        $\rightarrow$ Suppose that T on a finite-dimensional vector space is invertible. 

        FSOC, suppose that T is invertible, but zero is an eigenvalue. 

        $det([T] - 0I) = det([T]) = 0$ 

        Which means that A is not invertible, but that is a contradiction. So, zero must not be an eigenvalue in this case.

        $\leftarrow$ If zero is not an eigenvalue of T, then T is invertible. 
        Contrapositive: If T is not invertible, then zero is an eigenvalue. 

        To show that zero is eigenavalue, let us compute the determinant of the T - $\lambda I_n$  s.t. $\lambda = 0$
        
        $det([T] - 0I) = det(T) = 0$. 

        Hence, it follows that 0 is an eigenvalue of T when T is not invertible. 

        \qedsymbol

        \item Let T be an invertible linear operator. Prove that a scalar $\lambda$ is an eigenvalue of T if and only if $\lambda^{-1}$ is an eigenvalue of $T^{-1}$ 
        
        $\rightarrow $ Suppose that scalar $\lambda$ is an eigenvalue of T.

        Let v be a vector $\in$ an arbitrary finite dimensional space V s.t. $T(v) = \lambda v$

        WTS $\lambda^{-1}$ is an eigenvalue of $T^{-1}$. 

        $T^{-1}(T(v)) = T^{-1}(\lambda v)$

        $v = T^{-1}(\lambda v) = \lambda T^{-1}(v)$

        $\lambda^{-1}v = T^{-1}v$

        By definition, $\lambda^{-1}$ is an eigenvalue of $T^{-1}$. 

        $\leftarrow $ Suppose that $\lambda^{-1}$ is an eigenvalue of $T^{-1}$, then scalar $\lambda$ is an eigenvalue of T. 

        Let v be an eigenvector of an arbitrary finite dimensional space V s.t. $\lambda^{-1}v = T^{-1}v$. 

        $T^{-1}v = \lambda^{-1}v$

        $T(T^{-1}v) = T(\lambda^{-1}v)$

        $v = \lambda^{-1}T(v)$

        $\lambda v = T(v)$

        By definition, $\lambda$ is an eigenvalue of T.  \qedsymbol{}
        
    \end{enumerate}

    \item (Section 5.1 Problem 10) Prove that the eigenvalues of an upper triangular matrix M are the diagonal entries of M. 

    Let M be an $n \times n $ upper triangular matrix i.e. \\
    M = $\begin{pmatrix}
        m_{11} & m_{12} & .... &m_{1n} \\
        0 & m_{22} &............. \\
        .....&.....&m_{33}&..........\\
        0 & ...........& .... &m_{nn}
    \end{pmatrix}$

    To find the eigenvalues of M, we must find the determinant of $M - \lambda I_n$: 

    M = $\begin{pmatrix}
        m_{11} - \lambda & m_{12} & .... &m_{1n} \\
        0 & m_{22} - \lambda &............. \\
        .....&.....&m_{33} - \lambda&..........\\
        0 & ...........& .... &m_{nn} - \lambda
    \end{pmatrix}$

    Since $M - \lambda I_n$ is upper-triangular, the determinant is: 
    \begin{equation}
        (m_{11} - \lambda)(m_{22} - \lambda) .....(m_{nn} - \lambda) = 0
    \end{equation}

    And the roots are the diagonals of the upper triangular matrix M so we are done.

    \qedsymbol

    \item (Section 5.1 Problem 13a) Prove that similar matrices have the same characteristic polynomial.

    Similar matrices - Matrices with the same determinants and eigenvalues

    Let A be a matrix in $M_{n \times n}(F)$ for some n $\in \mathbb{N}$. 

    The characteristic polynomial of A is 
    \begin{equation}
        \chi_A(\lambda) = det(A - \lambda I_n) 
    \end{equation}

    To show that a similar matrix of A has the same characteristic polynomial as A, it is sufficient to show that it has the same characteristic polynomial. 

    Let B be a similar matrix to A and let Q be an invertible matrix s.t. $A = Q^{-1}BQ$

    \begin{equation}
        \chi_{A}(\lambda) = det(A - \lambda I_n) = det(Q^{-1}BQ - Q^{-1}\lambda I_nQ)
    \end{equation}
    \begin{equation}
        = det(Q^{-1} (B - \lambda I_n)Q) = det(Q^{-1})det(B - \lambda I_n)det(Q) = det(B - \lambda I_n)
    \end{equation}

    Thus, the similar matrix has the same characteristic polynomial as the original matrix. 
    
    \qedsymbol

    \item (Section 5.1 18a) 
    Let T be the linear operator on $M_{n \times n}(R)$ defined by T(A) = $A^T$. 

    Show that $\pm 1$ are the only eigenvalues of T.

    T(A) = $\lambda A$
    \begin{equation}
        A^T = \lambda A 
    \end{equation}
    \begin{equation}
        (A = \lambda A^T)^T \rightarrow A = \lambda A^T = \lambda^2 A
    \end{equation}

    If $A = \lambda^2 A$, then $\lambda$ can only be $\pm 1$. So we are done.
    \qedsymbol

    \item (Section 5.1 18c) 
    Let T be the linear operator on $M_{n \times n}(R)$ defined by T(A) = $A^T$. Find an ordered basis $\beta$ for $M_{2 \times 2}(R)$ s.t. $[T]_\beta$ is a diagonal matrix. 

    Let $\beta = \{\begin{pmatrix}
        1 & 0 \\
        0 & 0\\
    \end{pmatrix}, \begin{pmatrix}
        0 & 1 \\
        1 & 0\\
    \end{pmatrix}, \begin{pmatrix}
        0 & 0 \\
        0 & 1\\
    \end{pmatrix}, \begin{pmatrix}
        0 & 1 \\
        -1 & 0\\
    \end{pmatrix}\}$

    T($\begin{pmatrix}
        1 & 0 \\
        0 & 0\\
    \end{pmatrix}) = \begin{pmatrix}
        1 & 0 \\
        0 & 0\\
    \end{pmatrix} = 1\begin{pmatrix}
        1 & 0 \\
        0 & 0\\
    \end{pmatrix} \rightarrow \begin{pmatrix}
        1 \\
        0\\
        0\\
        0\\
    \end{pmatrix}$

    T($\begin{pmatrix}
        0 & 1 \\
        1 & 0\\
    \end{pmatrix}) = \begin{pmatrix}
        0 & 1 \\
        1 & 0\\
    \end{pmatrix} = 1\begin{pmatrix}
        0 & 1 \\
        1 & 0\\
    \end{pmatrix} \rightarrow \begin{pmatrix}
        0 \\
        1\\
        0\\
        0\\
    \end{pmatrix}$

    T($\begin{pmatrix}
        0 & 0 \\
        0 & 1\\
    \end{pmatrix}) = \begin{pmatrix}
        0 & 0 \\
        0 & 1\\
    \end{pmatrix} = 1\begin{pmatrix}
        0 & 0 \\
        0 & 1\\
    \end{pmatrix} \rightarrow \begin{pmatrix}
        0 \\
        0\\
        1\\
        0\\
    \end{pmatrix}$

    T($\begin{pmatrix}
        0 & 1 \\
        -1 & 0\\
    \end{pmatrix}) = \begin{pmatrix}
        0 & -1 \\
        1 & 0\\
    \end{pmatrix} = -1\begin{pmatrix}
        0 & 1 \\
        -1 & 0\\
    \end{pmatrix} \rightarrow \begin{pmatrix}
        0 \\
        0\\
        0\\
        -1\\
    \end{pmatrix}$

    $[T]_\beta = \begin{pmatrix}
        1 & 0 & 0 & 0 \\
        0 & 1 & 0 & 0 \\
        0 & 0 & 1 & 0 \\
        0 & 0 & 0 & -1 
    \end{pmatrix}$

    \item (Section 5.2 3c)
    V = $R^3$ and T is defined as $T\begin{pmatrix}
        a_1 \\
        a_2 \\
        a_3 \\
    \end{pmatrix} = \begin{pmatrix}
        a_2 \\
        -a_1 \\
        2a_3
    \end{pmatrix}$

    Let us use the standard basis of $R^3$: $\{\begin{pmatrix}
        1 \\
        0 \\
        0 \\
    \end{pmatrix}, \begin{pmatrix}
        0 \\
        1 \\
        0 \\
    \end{pmatrix}, \begin{pmatrix}
        0 \\
        0 \\
        1 \\
    \end{pmatrix} \}$

    T($\begin{pmatrix}
        1 \\
        0 \\
        0 \\
    \end{pmatrix}$) = $\begin{pmatrix}
        0 \\
        -1 \\
        0 \\
    \end{pmatrix} \rightarrow \begin{pmatrix}
        0 \\
        -1 \\
        0
    \end{pmatrix}$

    T($\begin{pmatrix}
        0 \\
        1 \\
        0 \\
    \end{pmatrix}$) = $\begin{pmatrix}
        1 \\
        0 \\
        0 \\
    \end{pmatrix} \rightarrow\begin{pmatrix}
        1 \\
        0 \\
        0
    \end{pmatrix}$

    T($\begin{pmatrix}
        0 \\
        0 \\
        1 \\
    \end{pmatrix}$) = $\begin{pmatrix}
        0 \\
        0 \\
        2 \\
    \end{pmatrix} \rightarrow \begin{pmatrix}
        0 \\
        0 \\
        2
    \end{pmatrix}$

    In matrix form, this matrix can be represented as: 
    $\begin{pmatrix}
        0 & 1 & 0 \\
        -1 & 0 & 0 \\
        0 & 0 & 2
    \end{pmatrix}$

    det(T - $\lambda I$) =  $\begin{pmatrix}
        -\lambda & 1 & 0 \\
        -1 & -\lambda & 0 \\
        0 & 0 & 2 - \lambda
    \end{pmatrix}$ = 0

    \begin{equation}
        (2 - \lambda)(\lambda^2 - (-1)) = (2 - \lambda)(\lambda^2 + 1) = 0
    \end{equation}

    T doesn't split over R so it is not diagonalizable. 

    \item (Section 5.2 3d)
    V = $P_2(R)$ and T is defined by $T(f(x)) = f(0) + f(1)(x + x^2)$

    Let $\beta$ be the standard ordered basis of $P_2(R)$. 
    
    T(1) = 1 + 1(x + $x^2)$ = 1 + x + $x^2$ $\rightarrow \begin{pmatrix}
        1 \\ 
        1 \\
        1\\
    \end{pmatrix}$

    T(x) = 0 + 1(x + $x^2)$ = x + $x^2$ $\rightarrow \begin{pmatrix}
        0 \\ 
        1 \\
        1\\
    \end{pmatrix}$

    T(x$^2$) = 0 + 1(x + $x^2)$ = x + $x^2$ $\rightarrow \begin{pmatrix}
        0 \\ 
        1 \\
        1\\
    \end{pmatrix}$

    So [T]$_\beta$ = $\begin{pmatrix}
        1 & 0 & 0 \\
        1 & 1 & 1 \\
        1 & 1 & 1 \\
    \end{pmatrix}$
    
    Characteristic polynomial: det([T]$_\beta$ - $\lambda$I) = 0
    
    $\begin{pmatrix}
        1 -\lambda & 0 & 0 \\
        1 & 1 - \lambda & 1 \\
        1 & 1 & 1 - \lambda \\
    \end{pmatrix} = (1 -\lambda)((1-\lambda)(1-\lambda) - 1 = (1 - \lambda)(\lambda^2 - 2\lambda) = 0$

    $\lambda(1 - \lambda)(\lambda - 2) = 0$

    $\lambda = 0, 1, 2$

    It is diagonalizable with D = $\begin{pmatrix}
        1 & 0 & 0 \\
        0 & 2 & 0 \\
        0 & 0 & 0
    \end{pmatrix}$

    $\lambda = 1$:

    $\begin{pmatrix}
        1 - 1 & 0 & 0 \\
        1 & 1 - 1 & 1 \\
        1 & 1 & 1 - 1\\
    \end{pmatrix} = \begin{pmatrix}
        0 & 0 & 0 \\
        1 & 0 & 1 \\
        1 & 1 & 0\\
    \end{pmatrix}$

    x + z = 0 \\
    x + y = 0 \\
    z is free

    z = 1, x = -1, y = 1

    $\begin{pmatrix}
        -1 \\
        1 \\
        1\\
    \end{pmatrix}$

    $\lambda = 2$:

    $\begin{pmatrix}
        1 - 2 & 0 & 0 \\
        1 & 1 - 2 & 1 \\
        1 & 1 & 1 - 2\\
    \end{pmatrix} = \begin{pmatrix}
        -1 & 0 & 0 \\
        1 & -1 & 1 \\
        0 & 0 & 0\\
    \end{pmatrix}$

    x = 0 \\
    - y + z = 0 \\
    z is free

    z = 1, x = 0, y = 1

    $\begin{pmatrix}
        0\\
        1 \\
        1\\
    \end{pmatrix}$

    $\lambda = 0$:

    $\begin{pmatrix}
        1 & 0 & 0 \\
        1 & 1 & 1 \\
        1 & 1 & 1 \\
    \end{pmatrix} = \begin{pmatrix}
        1 & 0 & 0 \\
        1 & 1 & 1 \\
        0 & 0 & 0\\
    \end{pmatrix}$

    x = 0 \\
    y + z = 0\\
    z is free

    z = 1, x = 0, y = -1

    $\begin{pmatrix}
        0\\
        -1 \\
        1\\
    \end{pmatrix}$

    Ordered basis can be $\{\begin{pmatrix}
        -1 \\
        1 \\
        1\\
    \end{pmatrix}, \begin{pmatrix}
        0\\
        1 \\
        1\\
    \end{pmatrix}, \begin{pmatrix}
        0\\
        -1 \\
        1\\
    \end{pmatrix}\}$ 

    and [T]$_\beta$ = $\begin{pmatrix}
        1 & 0 & 0 \\
        0 & 2 & 0 \\
        0 & 0 & 0
    \end{pmatrix}$
    
    
    \item (Section 5.2 7)

    A = $\begin{pmatrix}
        1 & 4 \\
        2 & 3 \\
    \end{pmatrix}$

    \begin{equation}
        A^k = QD^kQ^{-1}
    \end{equation}

    det(A - $\lambda$I) = $\begin{pmatrix}
        1 - \lambda & 4 \\
        2 & 3 - \lambda \\
    \end{pmatrix}$ = (1 - $\lambda$)(3 - $\lambda$) - (4 * 2) = \\
    3 - $\lambda - 3\lambda + \lambda^2$ - 8 = $\lambda^2 - 4\lambda - 5$ = 0

    $\lambda = -1, 5$

    $\lambda = -1$: \\
    det(A + I) = $\begin{pmatrix}
        1 + 1 & 4 \\
        2 & 3 + 1 \\
    \end{pmatrix} = \begin{pmatrix}
        2 & 4 \\
        2 & 4 \\
    \end{pmatrix}$

    2x + 4y = 0 \\
    x = -2, y = 1: $\begin{pmatrix}
        -2 \\
        1 \\
    \end{pmatrix}$

    $\lambda = 5$: \\
    det(A - 5I) = $\begin{pmatrix}
        1 - 5 & 4 \\
        2 & 3 - 5 \\
    \end{pmatrix} = \begin{pmatrix}
        -4 & 4 \\
        2 & -2 \\
    \end{pmatrix}$

    -x + y = 0 \\
    x = 1, y = 1
    $\begin{pmatrix}
        1 \\
        1 \\
    \end{pmatrix}$

    Q = $\begin{pmatrix}
        -2 & 1 \\
        1 & 1\\
    \end{pmatrix}$
    D = $\begin{pmatrix}
        -1 & 0 \\
        0 & 5
    \end{pmatrix}$

\begin{equation} Q^{-1} = 
    \frac{1}{det(Q)}
    \begin{pmatrix}
        2 & 1 \\
        1 & -1\\
    \end{pmatrix} = -\frac{1}{3}\begin{pmatrix}
        1 & -1 \\
        -1 & -2\\
    \end{pmatrix} = \begin{pmatrix}
        -\frac{1}{3} & \frac{1}{3} \\
        \frac{1}{3} & \frac{2}{3}\\
    \end{pmatrix}
\end{equation}

Answer:
\begin{equation}
    \begin{pmatrix}
        1 & 4 \\
        2 & 3 \\
    \end{pmatrix}^k = \begin{pmatrix}
        -2 & 1 \\
        1 & 1 \\
    \end{pmatrix}\begin{pmatrix}
        (-1)^k & 0 \\
        0 & 5^k \\
    \end{pmatrix}\begin{pmatrix}
        -\frac{1}{3} & \frac{1}{3} \\
        \frac{1}{3} & \frac{2}{3}\\
    \end{pmatrix}
\end{equation}
    
    \item (Section 5.2 8)
    Suppose that $A \in M_{n \times n}(F)$ has two distinct eigenvalues, $\lambda_1$ and $\lambda_2$ and that dim($E_{\lambda_{1}})$ = n - 1. Prove that A is diagonalizable.

    WTS dim($E_{\lambda_{1}})$ + dim($E_{\lambda_{2}})$ = n and A is diagonalizable. 

    Since $\lambda_1$ and $\lambda_2$ are distinct e-vals, the eigenbases of both eigenvalues are LI so it follows that the intersection of the two eigenspaces $E_{\lambda_{1}}$ and $E_{\lambda_{2}} = \{0\}$. 

    We know that dim($E_{\lambda_{1}})$ = n - 1, and dim($E_{\lambda_{2}}) \geq 1$. 

    Knowing that 

    dim($E_{\lambda_{1}})$ + $dim(E_{\lambda_{2}}) \leq n$. 

    Since dim($E_{\lambda_{1}})$ = n - 1, we can rewrite the equation as this: 
    \begin{equation}
        n - 1 + dim(E_{\lambda_{2}}) \leq n
    \end{equation}
    So $dim(E_{\lambda_{2}}) \leq 1$. 

    Since dim($E_{\lambda_{2}}) \geq 1$ and $dim(E_{\lambda_{2}}) \leq 1$, $$dim(E_{\lambda_{2}}) = 1$$ 

    Since the eigenvectors of both eigenspaces are LI, their union will form a basis for $M_{n \times n}(F)$, we can conclude that A is diagonalizable.

   
    \qedsymbol
    
    \item (Section 5.2 10)
    Let T be a linear operator on a finite-dimensional vector space V with the distinct eigenvalues $\lambda_1, \lambda_2, ..., \lambda_k$ and corresponding multiplicities $m_1, m_2, ...., m_k$. Suppose that $\beta$ is a basis for V such that [T]$_\beta$ is an upper triangular matrix. Prove that the diagonal entries of [T]$_\beta$ are $\lambda_1, \lambda_2, ..., \lambda_k$ and that each $\lambda_i$ occurs $m_i$ times. (1 $\leq$ i $\leq$ k).

    Proof:

    The characteristic polynomial of T is independent of the choice of basis $\beta$. For upper-triangular matrices, we know the determinant is the product of the diagonals so our characteristic polynomial is:
    \begin{equation}
        det([T]_\beta - \lambda I) = \chi_T (\lambda) = \prod_{i=1}^{k}(\lambda_i - \lambda)^{m_{i}}
    \end{equation}
    for i = 1, 2, 3,..., k.

    We know that the characteristic polynomial of T will split.
    If we set $\lambda = 0$, we get $\prod_{i=1}^{k}(\lambda)^{m_{i}}$. By the assumption we made that T has distinct eigenvalues $\lambda_1, \lambda_2, ..., \lambda_k$ with respective almus of $m_1, m_2, ...., m_k$, we can deduce that each eigenvalue $\lambda_i$ occurs exactly $m_i$ times on the diagonal entries of $[T]_\beta$. So we are done.
    \qedsymbol
    
    
\end{enumerate}

\end{document}
